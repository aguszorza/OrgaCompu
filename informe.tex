\documentclass[a4paper,10pt]{article}

\usepackage{graphicx}
\usepackage[ansinew]{inputenc}
\usepackage[spanish]{babel}
\usepackage{listings}
\usepackage{anysize} 

\marginsize{2cm}{2cm}{2cm}{2cm} 

\title{		\textbf{Informe TP0}}

\author{	Nombre y Apellido de Autor, \textit{Padr�n Nro. 00.000}                     \\
            \texttt{ direcci�n de e-mail }                                              \\[2.5ex]
            Nombre y Apellido de Autor, \textit{Padr�n Nro. 00.000}                     \\
            \texttt{ direcci�n de e-mail }                                              \\[2.5ex]
            Nombre y Apellido de Autor, \textit{Padr�n Nro. 00.000}                     \\
            \texttt{ direcci�n de e-mail }                                              \\[2.5ex]
            \normalsize{2do. Cuatrimestre de 2017}                                      \\
            \normalsize{66.20 Organizaci�n de Computadoras  $-$ Pr�ctica Martes}  \\
            \normalsize{Facultad de Ingenier�a, Universidad de Buenos Aires}            \\
       }
\date{}

\lstset{ %
  breakatwhitespace=true,
  breaklines=true,    
}

\begin{document}

\maketitle
\thispagestyle{empty}   % quita el n�mero en la primer p�gina


\pagebreak 


\section{Documentaci�n e implementaci�n}

El objetivo del trabajo es realizar un programa en lenguaje C que lea palabras de una archivo y guarde en otro archivo unicamente aquellas palabras que sean palindromos. Para ello, dividimos el programa en tres funciones. La funcion principal, main, se encargara de la logica de leer los parametros de entrada, el manejo de los archivos, y del bucle principal, que consiste en leer una palabra del archivo de entrada, comprobar si es palindromo y escribirla en el archivo de salida si corresponde. Si algun archivo no se puede abrir, o no se pasaron correctamente los parametros, el programa mostrara un mensaje de error en el archivo stderr y finalizara con un codigo de error.
Luego, habra una funcion leer\_palabra que se encarga de leer una palabra del archivo. Debido a las limitaciones de lo que se considera palabra, y a que no hay limitacion con respecto a cantidad de letras de una palabra, lo que hacemos es leer car�cter por car�cter, guardandolos en un vector alojado en memoria dinamica que se ira redimensionando a medida que sea necesario.
Por ultimo, la funcion es\_capicua, que se encarga de comprobar si la palabra es o no un palindromo, y devuelve un valor booleano seg�n corresponda.


\section{Comandos para compilacion}

Para compilar el programa, tanto en Linux como en NetBSD utilizamos el siguiente comando:
\newline \newline \$ gcc -Wall -o tp0 tp0.c 
\newline \newline Para obtener el codigo MIPS32 generado por el compilador utilizamos el siguiente comando en el sistema operativo NetBSD:
\newline \newline \$ gcc -Wall -O0 -S -mrnames tp0.c



\section{Pruebas}

Para probar el programa utilizamos un archivo de texto ?entrada.txt? que contiene un conjunto de palabras con combinaciones de letras, numeros y guiones y mezclando mayusculas y minusculas.
Luego tenemos otro archivo, ?resultado.txt? que es lo que se espera que devuelva el programa al ejecutarse con ese archivo de entrada. Para comprobar el resultado, utilizamos el siguiente comando:
\newline \newline \$ diff salida.txt resultado.txt
\newline \newline que si no muestra nada significa que ambos archivos son iguales, y que por lo tanto el programa funciona correctamente.



\subsection{Archivo 'entrada.txt'}

\lstinputlisting{entrada.txt}

\subsection{Archivo 'resultado.txt'}

\lstinputlisting{resultado.txt}

\section{Codigo fuente}
\lstinputlisting[language=C]{tp0.c}

\section{Codigo MIPS32}
\lstinputlisting{tp0.s}



\end{document}
\grid
